\documentclass[twoside]{article}
\usepackage{mmpr2017}

\begin{document}

\Russian
\title{Локальные модели для классификации объектов сложной структуры}
\author{Исаченко~Р.\,В.}{Исаченко Роман Владимирович$^{1, 2}$}{isa-ro@yandex.ru}
\author{Жариков~И.\,Н.}{Жариков Илья Николаевич$^{1, 2}$}{ilya250894@gmail.com}
\author{Бочкарёв~А.\,М.}{Бочкарёв Артём Максимович$^{1, 2}$\speaker}{artem.bochkarev@phystech.edu}
\organization{%
    $^1$Московский физико-технический институт\par
    $^2$Сколковский институт науки и технологий}
\maketitle

В работе рассматривается задача многоклассовой классификации объектов без явного признакового представления. 
Рассматриваемые данные содержат временные ряды ускорения, полученные акселерометра смартфона.
Задача состоит в предсказании вида физической активности человека по временному ряду.
При такой постановке временной ряд представляет собой объект сложной структуры без явного признакового описания.
В работе предлагается подход к генерации признаков временных рядов, рассматривая их как объекты сложной структуры.
В качестве признакового описания используются параметры локальных моделей.
Сгенерированные признаки позволяют получить приемлемое качество классификации и требуют умеренных вычислительных ресурсов.

Проблема классификации объектов сложной структуры разбивается на две несвязанные процедуры.
Первая извлекает информативные признаки. 
Вторая классифицирует объекты, используя порожденные признаки.

Данная работа нацелена на сравнение различных методов генерации признаков: экспертные функции, авторегрессионная модель и анализ сингулярного спектра.
Авторы предлагают новый метод порождения признаков.
Мы аппроксимируем сегменты временного ряда кубическими сплайнами. 
Сплайны генерируют гладкую кривую с достаточным качеством аппроксимации.

Эксперимент проводился на двух датасетов с акселерометра: WISDM, USC-HAD. 
Сравнивались качества упомянутых методов извлечения признаков и различных моделей классификации.

Работа поддержана грантом РФФИ \No\,16-07-01154.

\begin{thebibliography}{1}
\bibitem{author17first-word-of-the-title}
    \emph{Автор\;И.\,О.}
    Название статьи~//
    Название журнала,
    Город: Издательство, 2017.~--- С.\,5--25. %\\
    \url{http://jmlda.org/papers/doc/2017/no1/Author2017Title.pdf}.
\end{thebibliography}

\English
\title{Abstract template~--- MMPR"=2017}
\author{Author~N.}{Author Name$^{1, 2}$}{author\_email@site.ru}
\author{Coauthor~N.}{Coauthor Name$^2$\speaker}{coauthor@site.ru}
\organization{%
    $^1$City, Institution\par
    $^2$City, Institution}
\maketitle

The abstract should be concise and should present the aim of the work, essential results and conclusion.
The total length of the abstract should not exceed one page, that is approximately 1800 characters.
The abstract may contain plots, formulas and tables, in case if it helps to illustrate the results.
The abstract should not contain sections, lists, footnotes or floating images.
A reference to the grant can be provided in the last line.

The title should include full names, e-mails and institutional affiliations for all authors. The speaker should be marked using the command \verb|\speaker| (\speaker).
The bibliography must contain exactly one item~--- a reference to the full version of the article,
published or accepted for publication in a peer-reviewed journal,
or submitted to the peer-reviewed electronic journal~``Machine Learning and Data Analysis''
through the website of the journal \url{http://jmlda.org}.
If the full version of the article is published in an electronic journal, the URL for this article must be indicated.

This research is funded by RFBR, grant 00-00-00000.

\begin{thebibliography}{1}
\bibitem{author17first-word-of-the-title}
    \emph{Author\;N.}
    Paper name~//
    Journal,
    City:~Publisher, 2017.~--- p.\,5--25.
    \url{http://jmlda.org/papers/doc/2017/no1/Author2017Title.pdf}.
\end{thebibliography}

\end{document} 