\documentclass[twoside]{article}
\usepackage{mmpr2017}

\begin{document}

\Russian
\title{Локальные модели для классификации объектов сложной структуры}
\author{Исаченко~Р.\,В.}{Исаченко Роман Владимирович$^{1, 2}$}{isa-ro@yandex.ru}
\author{Жариков~И.\,Н.}{Жариков Илья Николаевич$^{1, 2}$}{ilya250894@gmail.com}
\author{Бочкарёв~А.\,М.}{Бочкарёв Артём Максимович$^{1, 2}$\speaker}{artem.bochkarev@phystech.edu}
\organization{%
    $^1$Московский физико-технический институт\par
    $^2$Сколковский институт науки и технологий}
\maketitle
	Данная работа посвящена предсказанию физической активности человека по сигналам акселерометра мобильного телефона. 
Сигнал акселерометра представляет собой временной ряд. 
В связи с большой размерностью описания объектов~---временных рядов и недостатком вычислительных мощностей возникает задача порождения признакового пространства. 
Авторы предлагают использовать параметры локальных моделей в качестве набора признаков. 
Эксперимент проводился на реальных данных акселерометра мобильного телефона: WISDM и USD-HAD.
В работе произведён анализ различных суперпозиций моделей порождения признаков и классификации.

Работа поддержана грантом РФФИ \No\,16-07-01154.

\begin{thebibliography}{1}
\bibitem{author17first-word-of-the-title}
    \emph{Автор\;И.\,О.}
    Название статьи~//
    Название журнала,
    Город: Издательство, 2017.~--- С.\,5--25. %\\
    \url{http://jmlda.org/papers/doc/2017/no1/Author2017Title.pdf}.
\end{thebibliography}

\English
\title{Abstract template~--- MMPR"=2017}
\author{Author~N.}{Author Name$^{1, 2}$}{author\_email@site.ru}
\author{Coauthor~N.}{Coauthor Name$^2$\speaker}{coauthor@site.ru}
\organization{%
    $^1$City, Institution\par
    $^2$City, Institution}
\maketitle

The abstract should be concise and should present the aim of the work, essential results and conclusion.
The total length of the abstract should not exceed one page, that is approximately 1800 characters.
The abstract may contain plots, formulas and tables, in case if it helps to illustrate the results.
The abstract should not contain sections, lists, footnotes or floating images.
A reference to the grant can be provided in the last line.

The title should include full names, e-mails and institutional affiliations for all authors. The speaker should be marked using the command \verb|\speaker| (\speaker).
The bibliography must contain exactly one item~--- a reference to the full version of the article,
published or accepted for publication in a peer-reviewed journal,
or submitted to the peer-reviewed electronic journal~``Machine Learning and Data Analysis''
through the website of the journal \url{http://jmlda.org}.
If the full version of the article is published in an electronic journal, the URL for this article must be indicated.

This research is funded by RFBR, grant 00-00-00000.

\begin{thebibliography}{1}
\bibitem{author17first-word-of-the-title}
    \emph{Author\;N.}
    Paper name~//
    Journal,
    City:~Publisher, 2017.~--- p.\,5--25.
    \url{http://jmlda.org/papers/doc/2017/no1/Author2017Title.pdf}.
\end{thebibliography}

\end{document} 