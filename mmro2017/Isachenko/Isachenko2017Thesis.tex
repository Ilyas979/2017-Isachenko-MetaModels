\documentclass[twoside]{article}
\usepackage{mmpr2017}

\begin{document}

\Russian
\title{Локальные модели для классификации объектов сложной структуры}
\author{Исаченко~Р.\,В.}{Исаченко Роман Владимирович$^{1, 2}$}{isa-ro@yandex.ru}
\author{Жариков~И.\,Н.}{Жариков Илья Николаевич$^{1, 2}$}{ilya250894@gmail.com}
\author{Бочкарёв~А.\,М.}{Бочкарёв Артём Максимович$^{1, 2}$\speaker}{artem.bochkarev@phystech.edu}
\organization{%
    $^1$Московский физико-технический институт\par
    $^2$Сколковский институт науки и технологий}
\maketitle

В работе рассматривается задача классификации объектов без явного признакового представления. 
Данные содержат временные ряды ускорения, полученные с акселерометра смартфона.
Задача состоит в предсказании вида физической активности человека по временному ряду.
Временной ряд представляет собой объект сложной структуры без явного признакового описания.
В работе предлагается подход к генерации признаков временных рядов, рассматривая их как объекты сложной структуры.
В качестве признакового описания используются параметры локальных моделей.
Сгенерированные признаки позволяют получить приемлемое качество классификации и требуют умеренных вычислительных ресурсов.

Проблема классификации объектов сложной структуры разбивается на две несвязанные процедуры.
Первая извлекает информативные признаки. 
Вторая классифицирует объекты, используя порожденные признаки.

Данная работа нацелена на сравнение различных методов генерации признаков: экспертные функции, авторегрессионная модель и анализ сингулярного спектра.
Авторы предлагают новый метод порождения признаков.
Сегменты временного ряда аппроксимируются кубическими сплайнами. 
Сплайны генерируют гладкую кривую с достаточным качеством аппроксимации.

Эксперимент проводился на двух датасетах с акселерометра: WISDM, USC-HAD. 
Сравнивались качества упомянутых методов извлечения признаков и различных моделей классификации.

Работа поддержана грантом РФФИ \No\,16-07-01154.

\begin{thebibliography}{1}
\bibitem{Isachenko17Local}
    \emph{Исаченко\;Р.\,В., Жариков\;И.\,Н., Бочкарёв\;А.\,М.}
    Локальные модели для классификации объектов сложной структуры~//
    Journal of Machine Learning and Data Analysis, 2016.~--- \No.\,13. %\\
    \url{http://jmlda.org/papers/doc/2016/no1/Isachenko2016MetricLearning.pdf}.
\end{thebibliography}

\English
\title{Local models for classification of complex structured objects}
\author{Isachenko~R.}{Roman Isachenko$^{1, 2}$}{isa-ro@yandex.ru}
\author{Zharikov~I.}{Ilya Zharikov$^{1, 2}$}{ilya250894@gmail.com}
\author{Bochkarev~A.}{Artem Bochkarev$^{1, 2}$\speaker}{artem.bochkarev@phystech.edu}
\organization{%
    $^1$Moscow Institute of Physics and Technology\par
    $^2$Skolkovo Instutute of Science and Technology}
\maketitle

The article investigates a problem of multiclass classification of objects without explicit feature set.
The data which were used are time series from smartphone accelerometer.
The problem is to predict human activity, using acceleration time series.
In this problem statement, time series is a complex structured object without explicit feature set.
The paper proposes method of feature generation for complex structured objects.
The features will correspond to parameters of local approximation models.
Features generated in such way allow decent classification quality and require low computational resources.

The problem of classification of complex structured objects is divided in two independent procedures.
First, informative features are extracted and then the classification model is built on these features.

This paper goal is to compare different approaches to feature generation: expert functions, autoregressive model and singular spectrum analysis.
The new method of feature generation is proposed.
This segments of time series are approximated with cubic splines, which generate smooth curve with sufficient quality of approximation.

The experiment was conducted on two real datasets: WISDM and USC-HAD.\\
The experiment shows comparison of different feature generation procedures and classification models.

This research is funded by RFBR, grant 16-07-01154.

\begin{thebibliography}{1}
\bibitem{Isachenko17LocalEng}
    \emph{Isachenko\;R., Zharikov\;I., Bochkarev\;A.}
    Local models for classification of complex structured objects~//
    Journal of Machine Learning and Data Analysis, 2016.~--- No\,13. %\\
    \url{http://jmlda.org/papers/doc/2016/no1/Isachenko2016MetricLearning.pdf}.
\end{thebibliography}

\end{document} 